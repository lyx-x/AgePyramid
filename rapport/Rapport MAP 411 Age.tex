%----------------------------------------------------------------------------------------
%	PACKAGES AND OTHER DOCUMENT CONFIGURATIONS
%----------------------------------------------------------------------------------------

\documentclass[paper=a4, french]{scrartcl} % A4 paper and 11pt font size

\usepackage[utf8]{inputenc} 
\usepackage[T1]{fontenc} % Use 8-bit encoding that has 256 glyphs
\usepackage{fourier} % Use the Adobe Utopia font for the document - comment this line to return to the LaTeX default
\usepackage[francais]{babel} % English language/hyphenation
\usepackage{amsmath,amsfonts,amsthm} % Math packages
\usepackage{envmath}
\usepackage{lipsum} % Used for inserting dummy 'Lorem ipsum' text into the template

\usepackage{sectsty} % Allows customizing section commands
\allsectionsfont{\centering \normalfont\scshape} % Make all sections centered, the default font and small caps

\usepackage{fancyhdr} % Custom headers and footers
\pagestyle{fancyplain} % Makes all pages in the document conform to the custom headers and footers
\fancyhead{} % No page header - if you want one, create it in the same way as the footers below
\fancyfoot[L]{} % Empty left footer
\fancyfoot[C]{} % Empty center footer
\fancyfoot[R]{\thepage} % Page numbering for right footer
\renewcommand{\headrulewidth}{0pt} % Remove header underlines
\renewcommand{\footrulewidth}{0pt} % Remove footer underlines
\setlength{\headheight}{13.6pt} % Customize the height of the header

\numberwithin{equation}{section} % Number equations within sections (i.e. 1.1, 1.2, 2.1, 2.2 instead of 1, 2, 3, 4)
\numberwithin{figure}{section} % Number figures within sections (i.e. 1.1, 1.2, 2.1, 2.2 instead of 1, 2, 3, 4)
\numberwithin{table}{section} % Number tables within sections (i.e. 1.1, 1.2, 2.1, 2.2 instead of 1, 2, 3, 4)

\setlength\parindent{0pt} % Removes all indentation from paragraphs - comment this line for an assignment with lots of text

%----------------------------------------------------------------------------------------
%	TITLE SECTION
%----------------------------------------------------------------------------------------

\newcommand{\horrule}[1]{\rule{\linewidth}{#1}} % Create horizontal rule command with 1 argument of height

\title{	
\normalfont \normalsize 
\textsc{Ecole Polytechnique} \\ [25pt] % Your university, school and/or department name(s)
\horrule{0.5pt} \\[0.4cm] % Thin top horizontal rule
\huge Pyramide des Ages \\ % The assignment title
\horrule{2pt} \\[0.5cm] % Thick bottom horizontal rule
}

\author{Zhixing CAO, Yuxiang LI} % Your name

\date{\normalsize\today} % Today's date or a custom date

\begin{document}

\setlength\parindent{12pt}

\maketitle % Print the title

% + - = ! / ( ) [ ] < > | ' :

%----------------------------------------------------------------------------------------
%	Introduction
%----------------------------------------------------------------------------------------

\section{Généralité}
Dans ce sujet, nous avons étudié l\rq{}évolution de la population en faisant intervenir le taux de fécondité (qui influence directement le taux de natalité) et le taux de mortalité. Pour ce faire, nous avons utilisé la méthode de différences finies, nous analyserons théoriquement la stabilité de notre schéma dans ce rapport. Pour visualiser son comportement, consultez le programme que nous avons écrit en Python.

%----------------------------------------------------------------------------------------
%	Prise en main du modèle
%----------------------------------------------------------------------------------------

\section{Prise en main du modèle}

Rappelons d\rq{}abord les notations que nous utilisons : $a$ l\rq{}âge ; $t$ le temps ; $\rho(a,t)$ la densité de la population ; $P(t)$ la population ; $\mu(a,P)$ le taux de mortalité ; $\beta(a)$ le taux de fécondité et $N(t)$ le nombre de nouveau-nés.

L\rq{}évolution de la population peut dorénavant s\rq{}interprêter avec les équations suivantes : 

\begin{align}
\frac{\partial \rho}{\partial t}(a,t) +  \frac{\partial \rho}{\partial a}(a,t) + \mu(a,P(t))\rho(a,t) & = 0 \label{eq:1}\\
\rho(0,t) & = N(t) \label{eq:2}\\
P(t) & = {\int_{0}^{+\infty}\rho(a,t)da} \label{eq:3}\\
N(t) & = \int_{0}^{+\infty}\beta(a)\rho(a,t)da \label{eq:4}
\end{align}

%------------------------------------------------

\paragraph{\textbf{Question 1.}}
~\\

Dans le cas où $\mu$ et $\beta$ sont constantes, nous avons :

\begin{equation*}
\begin{aligned}
{dP \over dt}(t)  & =  {\int_{0}^{+\infty}{d\rho \over dt}(a,t)da} \\
& =  {\int_{0}^{+\infty} - \frac{\partial \rho}{\partial a}(a,t) - \mu(a,P(t))\rho(a,t) da} \\
& =  -\mu P(t) - \rho (+ \infty,t) + \rho (0,t) \\
& = -\mu P(t) + \beta P(t) \\
\end{aligned}
\end{equation*}

Nous avons obtenu une équation différentielle linéaire constante d\rq{}ordre 1 pour $P(t)$ :

\begin{equation}
\fbox{${dP \over dt}(t) = (\beta - \mu)P(t)$}
\end{equation}

La résolution de cette équation nous donne :

\begin{equation}
\fbox{$P(t) = P(0) e^{(\beta - \mu)t}$}
\end{equation}

Dans cette situation-là, si $\beta > \mu$, on a plus de naissance que la mort, la population explose ; si $\beta < \mu$, la population disparaîtra ; seule le cas où $\beta = \mu$ permet d\rq{}avoir une population stable.

%------------------------------------------------

\paragraph{\textbf{Question 2.}}
~\\

Dans cette question, le taux de mortalité dépend de $P$, $\mu (P) = kP(t)$. On pourra remplacé $\mu$ par cette nouvelle formule dans $(2.5)$, car $\mu$ est toujours indépendant de $a$ et n\rq{}intervient pas dans l\rq{}intégrale. D\rq{}où on a une nouvelle équation différentielle :

\begin{equation*}
{dP \over dt}(t) = [\beta - \mu(t)]P(t)
\end{equation*}

Ou bien : 

\begin{equation}
\fbox{${dP \over dt}(t) = \beta P(t) - k{P(t)}^2$}
\end{equation}

Résolvons maintenant cette équation : 

\begin{equation*}
\begin{aligned}
\frac{d(P-\frac{\beta}{2k})}{{(P-\frac{\beta}{2k})}^2-\frac{\beta^2}{4k^2}}(t)  & =  -kdt \\
\frac{k}{\beta}(\frac{1}{P-\frac{\beta}{k}}-\frac{1}{P})d(P-\frac{\beta}{2k})& =  -kdt \\
ln(\frac{P}{P-\frac{\beta}{k}}) & =  \beta t + ln(\frac{P(0)}{P(0)-\frac{\beta}{k}})\\
\frac{P}{P-\frac{\beta}{k}} & = \frac{P(0)}{P(0)-\frac{\beta}{k}}e^{\beta t} \\
\end{aligned}
\end{equation*}

Finalement, nous trouvons en posant $C=\frac{P(0)}{P(0)-\frac{\beta}{k}}$ :

\begin{equation}
\fbox{$P(t) = \frac{\frac{\beta}{k}Ce^{\beta t}}{1-Ce^{\beta t}}$}
\end{equation}

Ce résultat montre bien que plus la population dépend positivement du taux de fécondité $\beta$ et négativement du taux de mortalité $\mu$. 

%------------------------------------------------

\paragraph{\textbf{Question 3.a.}}
~\\

Supposons que $\mu$ dépend de $a$, on a, puisqu\rq{}on confond $a$ et $t$ ici une équation différentielle pour $\phi(t) = \rho(t,t)$ :

\begin{equation}
\fbox{${d\phi \over dt}(t)  =  {d\rho \over dt}(t) = -\mu(t)\rho(t)$}
\end{equation}

\paragraph{\textbf{Question 3.b.}}
~\\

En résolvant l\rq{}équation précédente, on trouve :

\begin{equation*}
\begin{aligned}
ln(\rho(t)) - ln(\rho(0)) & = -\int_{0}^{t}\mu(a)da \\
\rho(t)& =  \rho(0)e^{-\int_{0}^{T}\mu(a)da} \\
\rho(A)& =  \rho(0)e^{-\int_{0}^{A}\mu(a)da} 
\end{aligned}
\end{equation*}

Utilisons l\rq{}hypothèse que ${\int_{0}^{A}\mu(a)da} = +\infty$, nous pouvons dire que $\rho(A) = 0$, ce qui implique que quite à translater les coordonnées en temps, la génération des nouveau-nés à $t = 0$ disparaîtra toujours à l\rq{}âge de $A$ ans. Personne ne peut vivre à l\rq{}âge $A$ puisse que tout le monde doit être né au moment où il avait $0$ ans.

%----------------------------------------------------------------------------------------
%	Discrétisation numérique
%----------------------------------------------------------------------------------------

\section{Discrétisation numérique}
Rappelons le schéma de discrétisation : 
\begin{align}
P^n & = \Delta{\sum_{i=0}^{N-1}\rho_i^n} \label{eq:5}\\
N^n & = \Delta\sum_{i=0}^{N-1}\beta(i\Delta a)\rho_i^n\label{eq:6}\\
\frac{\rho_{i}^{n+1}-\rho_{i}^{n}}{\Delta t} + \frac{\rho_{i}^{n}-\rho_{i-1}^{n}}{\Delta t} + \mu(i\Delta a,P^n)\rho_{i-1}^{n} & = 0 \ \ \ \ \ si\ i = 1,...,N \label{eq:7}\\
\frac{\rho_{0}^{n+1}-\rho_{0}^{n}}{\Delta t} + \frac{\rho_{0}^{n}-N^n}{\Delta a}& = 0 \label{eq:8}
\end{align}

%------------------------------------------------

\paragraph{\textbf{Question 4.a.}}
~\\

Etudions le schéma $(3.3)$ dans le cas de conditions aux limites périodiques en $a$. Ecrivons le schéma d\rq{}une autre manière : 

\begin{equation}
\rho_{i}^{n+1} = \rho_{i-1}^{n} (\frac{\Delta t}{\Delta a} - \mu(i\Delta a, P^n)\Delta t) + \rho_{i}^{n} (1-\frac{\Delta t}{\Delta a})
\end{equation}

Si on suppose que $\mu(a)$ positif et d\rq{}intégrale non nulle sur l\rq{}intervalle de périodicité, on sait qu\rq{}il exite pour $\Delta a$ petit, un $i^*$ tel que $\mu(i^*\Delta a) > 0$, d\rq{}après la formule ci-dessus, $\rho_{i^*}^{n}$ tend vers $0$ quand $n$ tend vers $+\infty$, car la somme des coefficients de cette combinaision est strictement inférieure à $1$.

%------------------------------------------------

\paragraph{\textbf{Question 4.b.}}
~\\

La condition de stabilité $l^\infty$ vient d\rq{}une combinaison convexe. Dans la formule $(3.5)$, nous voyons que pour que la combinaison soit convexe, il faut que :

\begin{equation*}
\begin{aligned}
(\frac{\Delta t}{\Delta a} - \mu(i\Delta a, P^n)\Delta t) + (1-\frac{\Delta t}{\Delta a}) & = 1 \\
\frac{\Delta t}{\Delta a} - \mu(i\Delta a, P^n)\Delta t & \geq 0 \\
1-\frac{\Delta t}{\Delta a} & \geq 0 
\end{aligned}
\end{equation*}

Soit :

\begin{equation}
\fbox{$\Delta t \leq \Delta a \leq \frac{1}{Max(\mu)}$}
\end{equation}

Cette même condition peut aussi garantir que $\rho_{i}^{n+1}\in[0,\max_{i}\rho_{i}^{n}]$.

%------------------------------------------------

\paragraph{\textbf{Question 5.}}
~\\

La matrice de transition s\rq{}écrit :
\begin{equation}
T_{N+1,N+1} =
\begin{pmatrix}
1-\frac{\Delta t}{\Delta a}+\beta(0) & \beta(\Delta a) & \beta(2\Delta a) & \cdots & \beta((N-1)\Delta a) & 0 \\
\frac{\Delta t}{\Delta a} - \mu(0, P^n) & 1-\frac{\Delta t}{\Delta a}+\beta(0) & 0 & \cdots & \cdots & 0\\
0  & \frac{\Delta t}{\Delta a} - \mu(\Delta a, P^n)  & \ddots & \ddots  & & \vdots\\
\vdots & 0 & \ddots & \ddots&0&\vdots\\
\vdots & & \ddots  & \ddots&1-\frac{\Delta t}{\Delta a}&0\\
0 & \cdots & \cdots & 0 & \frac{\Delta t}{\Delta a} - \mu(N\Delta a, P^n) & 1-\frac{\Delta t}{\Delta a}
\end{pmatrix}
\end{equation}

On remarque que cette matrice est légèrement différente des matrices habituelles, car elle change de valeur d\rq{}un étape à l\rq{}autre, ce qui nécessite un calcul supplémentaire pour effectuer une simulation.

En faisant une simulation numérique, nous voyons que $\Delta t \leq \Delta a$ correspond à un schéma stable mais ne respecte pas le principe du maximum discret dû au taux de fécondité (on change les conditions limites) et quant au cas où $\Delta t > \Delta a$, nous avons un schéma instable, ce qui est propre au schéma. On constate en même temps que si on prend $\Delta t = \frac{\Delta a}{10}$, le temps de calcul est plus long, mais la convergence se fait plus vite.

Dans le cas où $\Delta t = \Delta a$, le schéma s\rq{}écrit : 

\begin{equation}
\rho_{i}^{n+1} - \rho_{i-1}^{n}+\mu(i\Delta t, P^n)\rho_{i-1}^{n} = 0
\end{equation}

Et la matrice de transition aura son diagonale presque nulle, sauf le premier terme qui dépend de $\beta$.

%------------------------------------------------

\paragraph{\textbf{Question 6.}}
~\\

Si nous prenons un taux de mortalité de la forme :

\begin{equation}
\mu(a) = \frac{1}{A-a}
\end{equation}

Le schéma devient instable quand on raffine en $\Delta a$, parce que dans le schéma précédent, on demande $\mu(N\Delta a)$ qui pourra ne pas être définie. Il faut donc modifier le schéma et faire $\mu((i-1)\Delta a,P^n)\rho_{i-1}^{n}$.

Mais avec le taux de mortalité et de fécondité ainsi définis, la population explose tout le temps, on pourrait modifier la formule de $\mu$ en tenant compte de la population : 

\begin{equation}
\mu^*(a) = \frac{P}{A-a}
\end{equation}

%----------------------------------------------------------------------------------------

\end{document}