%----------------------------------------------------------------------------------------
%	PACKAGES AND OTHER DOCUMENT CONFIGURATIONS
%----------------------------------------------------------------------------------------

\documentclass[paper=a4, french]{scrartcl} % A4 paper and 11pt font size

\usepackage[utf8]{inputenc} 
\usepackage[T1]{fontenc} % Use 8-bit encoding that has 256 glyphs
\usepackage{fourier} % Use the Adobe Utopia font for the document - comment this line to return to the LaTeX default
\usepackage[francais]{babel} % English language/hyphenation
\usepackage{amsmath,amsfonts,amsthm} % Math packages
\usepackage{envmath}
\usepackage{lipsum} % Used for inserting dummy 'Lorem ipsum' text into the template

\usepackage{sectsty} % Allows customizing section commands
\allsectionsfont{\centering \normalfont\scshape} % Make all sections centered, the default font and small caps

\usepackage{fancyhdr} % Custom headers and footers
\pagestyle{fancyplain} % Makes all pages in the document conform to the custom headers and footers
\fancyhead{} % No page header - if you want one, create it in the same way as the footers below
\fancyfoot[L]{} % Empty left footer
\fancyfoot[C]{} % Empty center footer
\fancyfoot[R]{\thepage} % Page numbering for right footer
\renewcommand{\headrulewidth}{0pt} % Remove header underlines
\renewcommand{\footrulewidth}{0pt} % Remove footer underlines
\setlength{\headheight}{13.6pt} % Customize the height of the header

\numberwithin{equation}{section} % Number equations within sections (i.e. 1.1, 1.2, 2.1, 2.2 instead of 1, 2, 3, 4)
\numberwithin{figure}{section} % Number figures within sections (i.e. 1.1, 1.2, 2.1, 2.2 instead of 1, 2, 3, 4)
\numberwithin{table}{section} % Number tables within sections (i.e. 1.1, 1.2, 2.1, 2.2 instead of 1, 2, 3, 4)

\setlength\parindent{0pt} % Removes all indentation from paragraphs - comment this line for an assignment with lots of text

%----------------------------------------------------------------------------------------
%	TITLE SECTION
%----------------------------------------------------------------------------------------

\newcommand{\horrule}[1]{\rule{\linewidth}{#1}} % Create horizontal rule command with 1 argument of height

\title{	
\normalfont \normalsize 
\textsc{Ecole Polytechnique} \\ [25pt] % Your university, school and/or department name(s)
\horrule{0.5pt} \\[0.4cm] % Thin top horizontal rule
\huge Pyramide des Ages \\ % The assignment title
\horrule{2pt} \\[0.5cm] % Thick bottom horizontal rule
}

\author{Zhixing CAO, Yuxiang LI} % Your name

\date{\normalsize\today} % Today's date or a custom date

\begin{document}

\setlength\parindent{12pt}

\maketitle % Print the title

% + - = ! / ( ) [ ] < > | ' :

%----------------------------------------------------------------------------------------
%	Introduction
%----------------------------------------------------------------------------------------

\section{Généralité}
Dans ce sujet, nous avons étudié l\rq{}évolution de la population en faisant intervenir le taux de fécondité (qui influence directement le taux de natalité) et le taux de mortalité. Pour ce faire, nous avons utilisé la méthode de différences finies, nous analyserons théoriquement la stabilité de notre schéma dans ce rapport. Pour visualiser son comportement, consultez le programme que nous avons écrit en Python.

%----------------------------------------------------------------------------------------
%	Prise en main du modèle
%----------------------------------------------------------------------------------------

\section{Prise en main du modèle}

Rappelons d\rq{}abord les notations que nous utilisons : $a$ l\rq{}âge ; $t$ le temps ; $\rho(a,t)$ la densité de la population ; $P(t)$ la population ; $\mu(a,P)$ le taux de mortalité ; $\beta(a)$ le taux de fécondité et $N(t)$ le nombre de nouveau-nés.

L\rq{}évolution de la population peut dorénavant s\rq{}interprêter avec les équations suivantes : 

\begin{align}
\frac{\partial \rho}{\partial t}(a,t) +  \frac{\partial \rho}{\partial a}(a,t) + \mu(a,P(t))\rho(a,t) & = 0 \label{eq:1}\\
\rho(0,t) & = N(t) \label{eq:2}\\
P(t) & = {\int_{0}^{+\infty}\rho(a,t)da} \label{eq:3}\\
N(t) & = \int_{0}^{+\infty}\beta(a)\rho(a,t)da \label{eq:4}
\end{align}

%------------------------------------------------

\paragraph{\textbf{Question 1.}}
~\\

Dans le cas où $\mu$ et $\beta$ sont constantes, nous avons :

\begin{equation*}
\begin{aligned}
{dP \over dt}(t)  & =  {\int_{0}^{+\infty}{d\rho \over dt}(a,t)da} \\
& =  {\int_{0}^{+\infty} - \frac{\partial \rho}{\partial a}(a,t) - \mu(a,P(t))\rho(a,t) da} \\
& =  -\mu P(t) - \rho (+ \infty,t) + \rho (0,t) \\
& = -\mu P(t) + \beta P(t) \\
\end{aligned}
\end{equation*}

Nous avons obtenu une équation différentielle linéaire constante d\rq{}ordre 1 pour $P(t)$ :

\begin{equation}
\fbox{${dP \over dt}(t) = (\beta - \mu)P(t)$}
\end{equation}

La résolution de cette équation nous donne :

\begin{equation}
\fbox{$P(t) = P(0) e^{(\beta - \mu)t}$}
\end{equation}

Dans cette situation-là, si $\beta > \mu$, on a plus de naissance que la mort, la population explose ; si $\beta < \mu$, la population disparaîtra ; seule le cas où $\beta = \mu$ permet d\rq{}avoir une population stable.

%------------------------------------------------

\paragraph{\textbf{Question 1.}}
~\\



\begin{align} 
\begin{split}
(x+y)^3 	&= (x+y)^2(x+y)\\
&=(x^2+2xy+y^2)(x+y)\\
&=(x^3+2x^2y+xy^2) + (x^2y+2xy^2+y^3)\\
&=x^3+3x^2y+3xy^2+y^3
\end{split}					
\end{align}


Lorem ipsum dolor sit amet, consectetuer adipiscing elit. 
\begin{align}
A = 
\begin{bmatrix}
A_{11} & A_{21} \\
A_{21} & A_{22}
\end{bmatrix}
\end{align}
Aenean commodo ligula eget dolor. Aenean massa. Cum sociis natoque penatibus et magnis dis parturient montes, nascetur ridiculus mus. Donec quam felis, ultricies nec, pellentesque eu, pretium quis, sem.

%------------------------------------------------

\subsubsection{Heading on level 3 (subsubsection)}

\lipsum[3] % Dummy text

\paragraph{Heading on level 4 (paragraph)}

\lipsum[6] % Dummy text

%----------------------------------------------------------------------------------------
%	PROBLEM 2
%----------------------------------------------------------------------------------------

\section{Lists}

%------------------------------------------------

\subsection{Example of list (3*itemize)}
\begin{itemize}
	\item First item in a list 
		\begin{itemize}
		\item First item in a list 
			\begin{itemize}
			\item First item in a list 
			\item Second item in a list 
			\end{itemize}
		\item Second item in a list 
		\end{itemize}
	\item Second item in a list 
\end{itemize}

%------------------------------------------------

\subsection{Example of list (enumerate)}
\begin{enumerate}
\item First item in a list 
\item Second item in a list 
\item Third item in a list
\end{enumerate}

%----------------------------------------------------------------------------------------

\end{document}